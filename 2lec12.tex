\documentclass[a4article]{article}
\usepackage[left=3cm,right=3cm,top=2cm,bottom=2cm]{geometry}
\usepackage{graphicx} % Required for inserting images
\usepackage[english,russian]{babel}
\usepackage[T2A]{fontenc}
\input{lectures2/defines}


\title{Группы и алгебры Ли II}
\author{}
\date{}

\begin{document}

\maketitle

\section*{Лекция 12. Представления полупростых алгебр Ли}

\subsection*{Классификация конечномерных неприводимых представлений}
С прошлой лекции мы помним, что всякое представление старшего веса $\mu$ это фактор модуля Верма $M_{\mu}/W$ для некоторого $W \subset M_{\mu}$.
\begin{lemma}
\label{maximal_proper}
    Представление $M_{\mu}/W$ неприводимо тогда и только тогда, когда $W$ максимальное собственное подпредставление (то есть подпредставление, которое не содержится ни в каком другом собственном подпредставлении)
\end{lemma}
\begin{proof}
    В самом деле, допустим, $M_{\mu}/W$ неприводимо, а $W$ не является максимальным собственным подпредставлением. Тогда существует такое собственное подпредставление $W'$, что $W \subset W' \subset M_{\mu}$, а значит $W'/W \subset M_{\mu}/W$, что противоречит неприводимости. В другую сторону, допустим, $W$ - максимальный собственный подмодуль, а $M_{\mu}/W$ приводимо. Тогда найдется подпредставление в факторе: $V \subset M_{\mu}/W$. Рассмотрим подпространство $V+W$ в $M_{\lambda}$, где $V$ рассматривается как множество представителей в $M_{\mu}$. Оно является и подпредставлением, поскольку $\g.(V+W) \subset V+W$. Но $W \subset V+W$, значит $W$ не было максимальным.
\end{proof}
\begin{theorem}
    Для любого $\mu \in \mathfrak{h^*}$ существует единственное (с точностью до изоморфизма) неприводимое представление старшего веса $\mu$. Будем обозначать его $L_{\mu}$.
\end{theorem}
\begin{proof}
    Всякое собственное подпредставление $W \subset M_{\mu}$ допускает весовое разложение $W = \bigoplus W[\lambda]$, $W[\lambda] = W \cap M_{\mu}[\lambda]$, причем $W[\mu]=0$, иначе $W=M_{\mu}$. Пусть $W_{\mu}$ - сумма всех собственных подпредставлений. Поскольку $W_{\mu}[\mu]=0$, оно все еще собственное, а поскольку содержит все собственные подпредставления, то оно максимально. Тогда $L_{\mu}=M_{\mu}/W_{\mu}$ неприводимое, причем единственность следует из единственности максимального собственного подпредставления (которая следует из максимальности). 
\end{proof}

\begin{corollary}
    Для всякого неприводимого конечномерного представления $V$ существует $\lambda \in \mathfrak{h}^*$ такой, что $V \cong L_{\lambda}$.
\end{corollary}

\begin{proof}
    Всякое неприводимое конечномерное представление является представлением старшего веса, а значит является фактором модуля Верма $M_{\lambda}$ (по теореме с прошлой лекции) по максимальному собственному подмодулю (по лемме \ref{maximal_proper}).
\end{proof}

\begin{definition}
    Вес $\mu$ называется доминантным интегральным, если $\langle \mu, \alpha^{\vee} \rangle \in \mathbb{Z}_+$ для всех $\alpha \in R_+$. Множество всех доминантных интегральных весов назовем $P_+$
\end{definition}
\begin{lemma}
    $P_+ = P \cap \overline{C}_+$
\end{lemma}
\begin{proof}
    Сразу следует из определения.
\end{proof}
\begin{theorem}
\label{finitedim}
    Неприводимое представление старшего веса $L_{\mu}$ конечномерно тогда и только тогда, когда $\mu \in P_+$.
\end{theorem}
\begin{proof}
    Сперва докажем, что если $L_{\mu}$ конечномерно, то $\mu \in P_+$. Для этого рассмотрим $L_{\mu}$ как представление $(\mathfrak{sl}_2)_i$, образованной $\{e_{\alpha_i}, h_{\alpha_i}, f_{\alpha_i}\}$. Тогда $e_{\alpha_i}.v_{\mu}=0$, $h_{\alpha_i}.v_{\mu}=\langle \alpha_i, \mu\rangle v_{\mu}$. Тогда из конечномерности $L_{\mu}$ как представления $(\mathfrak{sl}_2)_i$ следует, что $\langle \alpha_i, \mu\rangle \in \mathbb{Z}_+$. Повторяя рассуждения для всех простых корней, получаем требуемое.

    Теперь докажем, что если $\mu \in P_+$, то $L_{\mu}$ конечномерно. Пусть $n_i = \langle \alpha_i^{\vee}, \mu \rangle$. Рассмотрим $v_{s_i.\mu}=f_i^{n_i+1}.v_{\mu}$, где $s_i.\mu = \mu - (n_i+1)\alpha_i$. Заметим, что тут мы по-новому определяем действие группы Вейля: мы рассмариваем не вес $\mu - n_i\alpha_i$, симметричный старшему относительно стенки $L_{\alpha_i}$, а следующий после него в соответствующем $(\mathfrak{sl}_2)_i$-подмодуле Верма. 

    Заметим теперь, что все $v_{s_i.\mu}$ - сингулярные, то есть для любых $j$ $e_{j}.v_{s_i.\mu}=0$. В самом деле, если $i \ne j$, то $[e_j, f_i]=0$ и $e_j.v_{\mu}=0$. А $e_i.v_{s_i.\mu}=0$ следует из теории представлений $(\mathfrak{sl}_2)_i$: $(\mathfrak{sl}_2)_i$-подмодуль Верма имеет старший вес $n_i$.

    Теперь рассмотрим подпредставление $M_i \subset M_{\mu}$, порожденное $v_{s_i.\mu}$. Оно не содержит старший вес, так что является собственным. Значит, $\sum M_i$ собственное. Рассмотрим $\tilde{L}_{\mu}=M_{\mu}/\sum M_i$. 
    \begin{proposition}
        $\tilde{L}_{\mu}$ конечномерно.
    \end{proposition}
    Представление $L_{\mu}$ - фактор по максимальному собственному подмодулю, значит является фактором $\tilde{L}_{\mu}$, а потому тоже конечномерно.
\end{proof}

Подводя итог, мы имеем
\begin{corollary}
    Для каждого $\mu \in P_+$ представление $L_{\mu}$ неприводимо, такие представления попарно неизоморфны и всякое неприводимое конечномерное представление изомофорно одному из них.
\end{corollary}
\subsection*{БГГ-резольвента}
В доказательстве последней теоремы мы ввели подпредставления $M_i \subset M_{\mu}$, порожденные сингулярными векторами $v_{s_i.\mu}$. 
\begin{lemma}
    Пусть $v \in M_{\mu}[\lambda]$ сингулярный, то есть такой, что $\mathfrak{n}_+.v=0$. Тогда подпредставление $M'$, порожденное $v$, изоморфно модулю Верма $M_{\lambda}$.
\end{lemma}
Представление $M'$ является представлением старшего веса $\lambda$, а значит по теореме с прошлой лекции имеется сюръективный морфизм $U\mathfrak{n}_ -\rightarrow M'$. Значит, нам достаточно проверить, что этот морфизм инъективен. Предположим, нашелся $u \in U\mathfrak{n}_-$ такой, что $uv=0$. Но мы знаем, что существует $u'\in U\mathfrak{n}_-$ такой, что $u'.v_{\mu}=v$. Таким образом, $uu'.v_{\mu}=0$. $M_\mu \cong U\mathfrak{n}_-$, поэтому это значит, что $uu'=0$, чего в $U\mathfrak{n}_-$ не бывает
\begin{theorem}
В предыдущих условиях
    \begin{enumerate}
        \item $M_i \cong M_{s_i.\mu}$, где $M_{s_i.\mu}$ - модуль Верма со старшим вектором $v_{s_i.\mu}=f_i^{n_i+1}.v_{\mu}$, $n_i = \langle\alpha_i^{\vee}, \mu \rangle$.
        \item $L_{\mu} = M_{\mu}/\sum M_i$
    \end{enumerate}
\end{theorem}
\begin{proof}
    Первое утверждение сразу следует из предыдущей леммы. 

    Рассмотрим $\tilde{L}_{\mu}=M_{\mu}/\sum M_i$. Это представление вполне приводимо, то есть $\tilde{L}_{\mu} = \bigoplus_{\lambda = \mu - \sum n_i\alpha_i} n_{\lambda}L_{\lambda}$. Поскольку $\dim\tilde{L}_{\mu}[\mu] = \dim L_{\mu}[\mu]$, $\tilde{L}_{\mu} = L_{\mu}\oplus \bigoplus_{\lambda \ne \mu} n_{\lambda}L_{\lambda}$. То есть старший вектор $\tilde{L}_{\mu}$ лежит в $L_{\mu}$, а значит $\tilde{L}_{\mu} \subset L_{\mu}$, значит $\tilde{L}_{\mu} = L_{\mu}$. 
\end{proof}

Нам хочется научиться считать размерности весовых подпространств в $L_{\mu}$. Размерности весовых подпространств в $M_{\mu}$ и $M_i$ мы можем посчитать по теореме $PBW$. Однако есть проблема: $\sum M_i$ - не прямая сумма, поскольку эти подмодули пересекаются. Мы имеем следующую точную последовательность:
\begin{equation}
    \bigoplus M_{s_i.\mu} \rightarrow M_{\mu}\rightarrow L_{\mu}\rightarrow 0,
\end{equation}
которая не является короткой точной, потому что первая стрелка не является инъективным морфизмом (опять же потому, что $M_i$ пересекаются). Однако можно написать длинную точную последовательность, в которой эти модули участвуют. Для этого определим новое действие группы Вейля по формуле $w.\lambda = w(\lambda+\rho)-\rho$, где $\rho = \frac{1}{2}\sum_{\alpha \in R_+}{\alpha}$.
\begin{lemma}
    Такое действие - это то же действие, которое было задано в доказательстве теоремы \ref{finitedim}
\end{lemma}
\begin{proof}
    Достаточно проверить, что $s_i.\lambda = \lambda - (\langle\alpha_i^{\vee}, \lambda \rangle+1)\alpha_i$
\end{proof}
\begin{theorem}(Резольвента Бернштейна-Гельфанда-Гельфанда)
    Пусть $\lambda \in P_+$. Тогда
    \begin{equation}
        0 \rightarrow M_{w_0.\mu} \rightarrow \ldots \rightarrow \bigoplus_{w \in W, l(w)=k} M_{w.\mu} \rightarrow \ldots \rightarrow \bigoplus_{i=1,\ldots, r} M_{s_i.\mu} \rightarrow M_{\mu} \rightarrow L_{\mu} \rightarrow 0
    \end{equation}
\end{theorem}
\begin{example}
    Рассмотрим $\mathfrak{sl}_2$. Тогда $\mathfrak{h}^* \cong \mathbb{C}$ с изоморфизмом, заданным формулой $\lambda \mapsto \langle h, \lambda \rangle$. Тогда $\alpha \mapsto 2$. Решетка весов определяется соотношением $\frac{2(\alpha, \lambda)}{(\alpha, \alpha)}=1/2\alpha\lambda=\lambda \in \mathbb{Z}$. Это соответствует нашему знанию о том, что неприводимые конечномерные представления нумеруются неотрицательными целыми числами. 

    Заметим, что $\rho=1/2\alpha=1$. Тогда 
    $s.\lambda = s(\lambda + 1)-1=-(\lambda+1)-1=-\lambda-2$, и в случае $\mu \in \mathbb{Z}_+$
    БГГ-резольвента имеет вид
    \begin{equation}
        0 \rightarrow M_{-\mu-2} \rightarrow M_{\mu} \rightarrow L_{\mu} \rightarrow 0,
    \end{equation}
    и $L_{\mu} = M_{\mu}/M_{-\mu-2}$.
\end{example}
\begin{example}
Рассмотрим $\mathfrak{sl}_3$. В этом случае БГГ-резольвента имеет вид
\begin{equation}
    0 \rightarrow M_{s_1s_2s_1.\mu}\xrightarrow{\phi_2} M_{s_1s_2.\mu}\oplus M_{s_2s_1.\mu} \xrightarrow{\phi_3} M_{s_1.\mu}\oplus M_{s_2.\mu} \xrightarrow{\phi_4} M_{\mu} \xrightarrow{\phi_5} L_{\mu} \rightarrow 0
\end{equation}
Для тривиального представления вложения соответствующих модулей Верма в $M_0$ показаны на рисунке. Сингулярные вектора имеют вид $$v_1 = v_{s_1.0}=f_1.v_0$$ $$v_2=v_{s_2.0}=f_2v_0$$ $$v_3 = v_{s_1s_2.0}=f_1^2f_2v_0$$ $$v_4=v_{s_2s_1.0}=f_2^2f_1v_0$$ $$v_5=v_{s_1s_2s_1.0}=f_1f_2^2f_1v_0=f_2f_1^2f_2v_0$$
    \begin{figure}[h!]
    \centering
    \includegraphics[width = 0.45\textwidth]{BGG.jpeg}
    \caption{Подмодули $M_{0}$, участвующие в БГГ резольвенте тривиального представления $\mathfrak{sl}_3$}
    \label{fig:enter-label}
\end{figure}

Морфизмы устроены так:
$\phi_2: v_5 \mapsto (v_5, -v_5)$, $\phi_3: (v_3, v_4) \mapsto (v_3+v_4, -v_3-v_4)$, $\phi_4: (v_1, v_2) \mapsto v_1+v_2$.

Единственное, что остается понять и что не очевидно из картинки, это как вложен модуль Верма, порожденный $v_3$, в модуль Верма, порожденный $v_1$, и как вложен модуль Верма, порожденный $v_4$, в модуль Верма, порожденный $v_2$. Это можно сделать с помощью соотношений Серра: $[f_1, [f_1,f_2]]=0$ и $[f_2, [f_2,f_1]]=0$. Тогда $v_3 = f_1^2f_2v_0=(2f_1f_2-f_2f_1)f_1v_0=(2f_1f_2-f_2f_1)v_1$, $v_4=f_2^2f_1v_0=(2f_2f_1-f_1f_2)f_2v_0=(2f_2f_1-f_1f_2)v_2$.

\end{example}
\end{document}