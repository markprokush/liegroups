\documentclass[a4article]{article}
\usepackage[left=3cm,right=3cm,top=2cm,bottom=2cm]{geometry}
\usepackage{graphicx} % Required for inserting images
\usepackage[english,russian]{babel}
\usepackage[T2A]{fontenc}
\input{lectures2/defines}


\title{Группы и алгебры Ли II}
\author{}
\date{}

\begin{document}

\maketitle

\section*{Лекция 10}

\subsection*{Решетки корней и весов}
На прошлой лекции мы столкнулись с решеткой корней $Q$ и решеткой весов $P$ системы корней $A_2$. Дадим определения этих решеток в случае произвольной системы корней $R$.

\begin{definition}
    Решетка корней $Q$ системы корней $R$ - это абелева группа в $E$, порожденная $R$. Иначе говоря, если $\Pi = \{\alpha_1, \ldots, \alpha_r\}$ набор простых корней $R$, то $Q = \bigoplus_{i=1}^{r}\mathbb{Z}\alpha_i$.
\end{definition}
Вспомним, что конечномерность неприводимого представления старшего веса $\mu$ приводила к условию $\mu(H_{ij})=\frac{2(e_i-e_j, \mu)}{(e_i-e_j, e_i-e_j)} \in \mathbb{Z}_+$. В случае $A_2$ из этого условия мы заключили, что $\mu$ лежит на решетке $\bigoplus_{i=1}^{r}\mathbb{Z}e_i$ и назвали ее решеткой весов $P$. 
\begin{definition}
    Решетка весов $P$ системы корней $R$ - это абелева группа в $E$, определенная следующим образом:
    \begin{equation}
        P = \{\lambda \in E| \frac{2(\lambda, \alpha)}{(\alpha, \alpha)} \in \mathbb{Z} \quad \forall \alpha \in R\}
    \end{equation}
\end{definition}
\begin{definition}
    Фундаментальные веса $\omega_i \in P$ - это веса, удовлетворяющие условию $\frac{2(\lambda_i, \alpha_j)}{(\alpha_j, \alpha_j)}=\delta_{ij}$.
\end{definition}
Фундаментальные веса образуют базис $P$:
\begin{equation}
    P = \bigoplus_{i=1}^{r}\mathbb{Z}\omega_i
\end{equation}
\begin{example}
    Фундаментальные веса решетки весов системы корней $A_2$ - это $\{e_1, e_1+e_2\}$.
\end{example}

Пусть имеется какая-то решетка в евклидовом пространстве $L \subset E$, то есть некоторая абелева подгруппа $E$ с числом образующих, равных размерности $E$.
\begin{definition}
    Два элемента $\alpha, \beta \in E$ конгруэнтны относительно решетки $L$, если $\alpha - \beta \in L$.
\end{definition}


\subsection*{Описание набора весовых подпространств, окончание}

На прошлой лекции мы установили, что:
\begin{enumerate}
    \item Всякое конечномерное представление $V$ $\mathfrak{sl}_3$ имеет старший вес;
    \item Старший вес $\mu$ неприводимого представления $V$ единственный, лежит на решетке весов, и $dim V[\mu]=1$;
    \item Все веса неприводимого представления $V$ заключены внутри шестиугольника с вершинами, полученными из $\mu$ отражениями относительно прямых $L_{e_i-e_j}$, причем если вес $\lambda$ лежит на стороне шестиугольника, то $dim V[\lambda]=1$.
\end{enumerate}

Нам осталось понять, какие из весов внутри шестиугольника присутствуют в весовом разложении. По теореме 2 предыдущей лекции в него могут входить только веса, конгруэнтные $\mu$, то есть такие $\lambda$, что $\lambda - \mu$ лежит на решетке корней. На самом деле каждый такой вес входит в разложение с ненулевой кратностью: возьмем любой вес $\beta$ на одном из ребер шестиугольника, выберем $w \in V[\beta]$ и подействуем на него $E_{ij}$ (выберем $E_{ij}$ так, чтобы вес $\beta + e_i - e_j$ был внутри шестиугольника). Мы получим неприводимое представление $\mathfrak{sl}_2$-тройки $\{E_{ij}, H_{ij}, E_{ji}\}$, веса которого расположены симметрично относительно прямой $L_{e_i-e_j}$. Это значит, что все слагаемые прямой суммы $\bigoplus_{k \in \mathbb{Z}_+}V[\beta + k(e_i-e_j)]$ ненулевые.

\begin{figure}[h!]
\centering
\includegraphics[width = 0.5\textwidth]{lectures2/weightsofrep.jpeg}
\caption{набор весов неприводимого представления со страшим весом $\alpha$}
\label{fig:enter-label}
\end{figure}

\begin{theorem}
    Пусть $V$ - неприводимое конечномерное представление $\mathfrak{sl}_3$. Тогда для некоторого $\mu$, лежащего на решетке весов, множество весов представления $V$ - это такие $\lambda \in P$, что они конгруэнтны $\mu$ и лежат внутри шестиугольника с вершинами, полученными отражениями относительно прямых $L_{e_i-e_j}$.
\end{theorem}

\subsection*{Примеры}

Простейший нетривиальный пример неприводимого представления - это стандартное представление $V = \mathbb{C}^3$. В стандартном базисе $\{v_1, v_2, v_3\}$ для любого элемента $h \in \mathfrak{h}$ $h.v_i = h_i v_i = e_i(h)v_i$, поэтому весовое разложение $V = \bigoplus_{i=1}^{3} V[e_i]$.
\begin{figure}[h!]
\centering
\includegraphics[width = 0.5\textwidth]{lectures2/fundamentalrep.jpeg}
\caption{Стандартное (фундаментальное) представление $V$}
\label{fig:enter-label}
\end{figure}

Следующий пример - двойственное представление $W=V^*$. Поскольку $\forall x \in \g$ $\langle x.v, u \rangle + \langle v, x.u \rangle = 0$, веса $V^*$ - это веса $V$, взятые с противоположным знаком.
\begin{figure}[h!]
\centering
\includegraphics[width = 0.5\textwidth]{lectures2/cofundamentalrep.jpeg}
\caption{Двойственное к стандартному представление $V^*$}
\label{fig:enter-label}
\end{figure}

Рассмотрим $W = Sym^2V$. Базис в этом случае - это мономы $v_i v_j$, причем $\mathbb{C}v_iv_j = W[e_i+e_j]$, $W = \bigoplus W[e_i+e_j]$. Представление $W$ неприводимо, поскольку все его весовые подпространства одномерны.

\begin{lemma}
    Представление $V_{n,0}=Sym^a V$ неприводимо.
\end{lemma}
\begin{proof}
    Следует из того, что $Sym^a V$ имеет старший вектор $v_1^a$, а каждое его весовое подпространство одномерно: вес $ke_1+le_2+me_3$, $k+l+m=a$ соответствует подпространству $\mathbb{C}v_1^kv_2^lv_3^m$ и только ему.
\end{proof}
\begin{corollary}
    Представление $V_{0,n}=Sym^a V^*$ неприводимо.
\end{corollary}
\begin{figure}[h!]
\centering
\includegraphics[width = 0.5\textwidth]{lectures2/symrep.jpeg}
\caption{$Sym^2V$}
\label{fig:enter-label}
\end{figure}

Рассмотрим $W = V \otimes V^*$. В этом случае весовые подпространства $W[e_i - e_j] = \mathbb{C}v_i \otimes v_j^*$, $W = \bigoplus W[e_i-e_j]$. Представление $W$ приводимо. В самом деле, мы можем рассмотреть морфизм представлений $i: V \otimes V^* \rightarrow \mathbb{C}$, заданный спариванием: $v \otimes u^* \mapsto \langle v, u^*\rangle$. Его ядро - это бесследовые операторы $\mathfrak{sl}(V)$: $i(\sum_{i,j} a_{ij}v_i\otimes v^*_j) = \sum_{i,j}a_{ij}\langle v_i, v_j^* \rangle = \sum_{i}a_{ii} = 0$. Но вообще-то ядро морфизма представлений - это тоже представление, причем в этом случае мы получили, что $\ker {i}$ - присоединенное представление, которое является неприводимым по доказанному на прошлой лекции. Так что $W = \mathbb{C} \oplus \mathfrak{sl}_3$.

\begin{figure}[h!]
\centering
\includegraphics[width = 0.5\textwidth]{lectures2/VVst_rep.jpeg}
\caption{$V\otimes V^*$}
\label{fig:enter-label}
\end{figure}

Теперь рассмотрим $W = Sym^2 V \otimes V^*$. Это представление имеет одномерные весовые подпространства $W[e_i + e_j - e_k] = \mathbb{C}v_iv_j \otimes v_k^*$ для $k \ne i, j$  и трехмерные весовые подпространства 
$W[e_i] = \mathbb{C}v_iv_j\otimes v_j^* \oplus \mathbb{C}v_iv_k\otimes v_k^*\oplus \mathbb{C}v_iv_i\otimes v_i^*)$. Представление $W$ снова приводимо. В самом деле, рассмотрим морфизм представлений $i: Sym^2 V \otimes V^* \rightarrow V$, снова заданный спариванием: $uv \otimes w \mapsto \langle u, w \rangle v+\langle v, w \rangle u$. Его ядро состоит из одномерных весовых подпространств $W[e_i+e_j-e_k]$, $k \ne i, j$ и двумерных весовых подпространств $W[e_i]$. Ядро морфизма $\ker{i}$ является подпредставлением. 
\begin{figure}[h!]
\centering
\includegraphics[width = 0.5\textwidth]{lectures2/finalexamprep.jpeg}
\caption{$Sym^2V\otimes V^*$}
\label{fig:enter-label}
\end{figure}

\newpage
\subsection*{Описание неприводимых представлений}

Нам осталось разобраться с двумя вопросами - классификацией неприводимых представлений и кратностью вхождения в них весовых подпространств.
\begin{theorem}
    Для всякой пары целых неотрицательных $a, b$ существует единственное неприводимое конечномерное представление $V_{a,b}$ со старшим весом $ae_1 - be_3$.
\end{theorem}
\begin{proof}
    Существование $V_{a,b}$ следует из того, что представление $Sym^a V \otimes Sym^b V^*$ содержит старший вектор веса $ae_1 - be_3$. Докажем единственность $V_{a,b}$. Предположим, нашлось два представления со старшим весом $\mu$, скажем, $V$ и $W$ со старшими векторами $v \in V[\mu]$ и $w \in W[\mu]$. Рассмотрим прямую сумму представлений $V \oplus W$. Вектор $(v, w) \in V \oplus W$ - старший с весом $\mu$. Рассмотрим неприводимое подпредставление $U \subset V\oplus W$, порожденное старшим вектором $(v, w)$. Теперь рассмотрим проекции $\pi_V: U \rightarrow V$ и $\pi_W: U \rightarrow W$. Это морфизмы неприводимых представлений, значит по лемме Шура $U \cong V$ и $U \cong W$, так что $V \cong W$.
\end{proof}

\begin{theorem}
    Пусть $b \le a$. Тогда имеется такое разбиение:
    \begin{equation}
        Sym^a V \otimes Sym^b V^* = \bigoplus_{i=0}^{b} V_{a-i, b-i}
    \end{equation}
\end{theorem}
\begin{remark}
    Сравним это с результатом для представлений $\mathfrak{sl}_2$:
    \begin{equation}
        V_a \otimes V_b = \bigoplus_{i=0}^{b} V_{a-b + 2i}
    \end{equation}
\end{remark}
\begin{proof}
    Сначала заметим, что все веса $Sym^a V \otimes Sym^b V^*$ лежат внутри шестиугольника, соответствующего весовой диаграмме представления $V_{a,b}$, так что набор весов у $Sym^a V \otimes Sym^b V^*$ и $\bigoplus_{i=0}^{b} V_{a-i, b-i}$ одинаковый. Весовая диаграмма устроена следующим образом: она состоит из $b$ шестиугольников $H_i$, $i=0, \ldots b-1$, с вершинами в точках $(a-i)e_1 - (b-i)e_3$, которые сменяются треугольниками $T_j$, $j = 1, \ldots, [(a-b)/3]$ с вершинами в точках $(a-b-3j)$.

    Размерность весовых подпространств в $Sym^a V \otimes Sym^b V^*$, соответствующих точкам на шестиугольнике $H_i$, равна $\frac{(i+1)(i+2)}{2}$, а размерность весовых подпространств, соответствующих точкам на треугольнике $T_j$, равна $\frac{(b+1)(b+2)}{2}$. 

    Пусть $\mu = (a-i)e_1-(b-i)e_3$.  Заметим, что отображение $E_{21}^m: [\mu] \rightarrow V[\mu+m(e_2-e_1)]$ инъективно, если вес $\mu+m(e_2-e_1)$ еще принадлежит весовой диаграмме, поскольку иначе существует вектор $v \in V[\mu]$ такой, что $E_{21}^m.v=0$, что невозможно из-за веса $v$. Но $dim(V[\mu])=dim(V[\mu+m(e_2-e_1)])$, поэтому отображение $E_{21}^m: V[\mu] \rightarrow V[\mu+m(e_2-e_1)]$ - изоморфизм. Как следствие, любой вектор $w \in V[\mu+m(e_2-e_1)]$ имеет вид $w = E_{21}^m.v$ для некоторого $v \in V[\mu]$, а значит $E_{12}.w \ne 0$. Таким образом, в весовое разложение $Sym^a V \otimes Sym^b V^*$ не могут входить никакие веса, кроме уже обозначенных. Нам остается проверить, что все обозначенные там есть.

    Проверим, что каждое весовое подпространство $V[(a-i)e_1-(b-i)e_3]$ содержит  старший вектор. Зададим общий вид вектора веса $(a-i)e_1-(b-i)e_3$: $v = \sum_{(i_1, i_2, i_3)}c_{(i_1, i_2, i_3)}v_1^{(a-i)}v^{(i_1, i_2, i_3)}\otimes v^{*(b-i)}v^{*(i_1, i_2, i_3)}$, где $v^{(i_1, i_2, i_3)}=v_1^{i_1}v_2^{i_2}v_3^{i_3}$, $i_1+i_2+i_3=i$. Найдем коэффициенты $c_{(i_1, i_2, i_3)}$, при которых он будет старшим, то есть $E_{12}.v=E_{23}.v=0$.
    Воспользуемся знанием о том, что \begin{equation}
        \begin{gathered}
            E_{12}.v_2=v_1\\
            E_{12}.v_1^*=-v_2^*
        \end{gathered}
    \end{equation}
    \begin{equation}
    \begin{gathered}
        E_{12}.v_1^{(a-i)}v^{(i_1, i_2, i_3)}\otimes v^{*(b-i)}v^{*(i_1, i_2, i_3)} = \\
        i_2v_1^{(a-i)}v^{(i_1+1, i_2-1, i_3)}\otimes v^{*(b-i)}v^{*(i_1, i_2, i_3)}-i_1v_1^{(a-i)}v^{(i_1, i_2, i_3)}\otimes v^{*(b-i)}v^{*(i_1-1, i_2+1, i_3)}
    \end{gathered}
    \end{equation}
    
    Поэтому чтобы $v$ был старшим, необходимо и достаточно, чтобы $i_2c_{(i_1, i_2, i_3)}=(i_1+1)c_{(i_1+1, i_2-1, i_3)}$ и 
    $i_3c_{i_1,i_2,i_3}=(i_2+1)c_{(i_1, i_2+1, i_3-1)}$. Отсюда видно, что $i_1!i_2!i_3!c_{(i_1, i_2, i_3)}$ не зависит от $i_1, i_2, i_3$, так что $c_{(i_1, i_2, i_3)}=c/i_1!i_2!i_3!$ для произвольного $c$. Таким образом, для каждого веса $(a-i)e_1-(b-i)e_3$ мы нашли старший вектор, а значит и неприводимое подпредставление $V_{a-i, b-i} \subset Sym^a V \otimes Sym^b V^*$.
\end{proof}
\end{document}