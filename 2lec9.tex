\documentclass[a4article]{article}
\usepackage[left=3cm,right=3cm,top=2cm,bottom=2cm]{geometry}
\usepackage{graphicx} % Required for inserting images
\usepackage[english,russian]{babel}
\usepackage[T2A]{fontenc}
\input{lectures2/defines}


\title{Группы и алгебры Ли II}
\author{}
\date{}

\begin{document}

\maketitle

\section*{Лекция 9}



\subsection*{Неприводимые представления $\mathfrak{sl}_2$, напоминание}
\begin{theorem}
    Пусть $n \in \mathbb{Z}_{\ge 0}$, $V_n = \langle v^0, v^1, \ldots, v^n \rangle$. Определим действие $\mathfrak{sl}(2, \mathbb{C})$ на $V_n$ по формулам
    \begin{equation}
        \begin{gathered}
            fv^k = (k+1)v^{k+1}, k < n, fv^n = 0\\
            hv^k = (n - 2k)v^k,\\
            ev^k = (n - k + 1)v^{k-1}, k > 0, ev^0 = 0
        \end{gathered}
        \label{eqs_Vn}
    \end{equation}
    Тогда $V_n$ - неприводимое. Мы будем называть его представлением старшего веса $n$. Любое неприводимое конечномерное представление $\mathfrak{sl}(2, \mathbb{C})$ изоморфно представлению старшего веса.
\end{theorem}
\begin{proposition}
    Пусть $V$ - тавтологическое представление $\mathfrak{sl}_2$. Тогда $S^n(V) \cong V_n$ как неприводимые представления $\mathfrak{sl}_2 (\C)$ (в обозначениях лекции).
\end{proposition}


\subsection*{Весовое разложение представлений $\mathfrak{sl}_3$}

Изучение конечномерных представлений алгебры $\mathfrak{sl}_3$ будет похоже на деятельность в случае $\mathfrak{sl}_2$, но добавятся ровно те концепции, которые потребуются нам в изучении конечномерных представлений произвольной полупростой алгебры Ли.

Первое, что мы сделали, когда начали изучать конечномерные представления $\mathfrak{sl}_2$, это выяснили, что всякое такое представление $V$ можно разложить в прямую сумму весовых подпространств: $V = \bigoplus V_{\lambda}$, где $\forall v \in V_{\lambda}$ $h.v = \lambda v$. Здесь мы поступим так же, вспомнив, что аналогом элемента $h$ выступает двумерная картановская подалгебра $\mathfrak{h}$. Все элементы $h \in \mathfrak{h}$ диагонализуемы и коммутируют, так что диагонализуемы одновременно. Поэтому верна такая
\begin{lemma}
    Любое конечномерное представление $V$ алгебры $\mathfrak{sl}_3$ имеет весовое разложение:
    \begin{equation}
        V = \bigoplus V[\lambda],
    \end{equation}
    где $\forall h \in \mathfrak{h}$ и $\forall v \in V[\lambda]$ $h.v = \lambda(h)v$, то есть $\lambda \in \mathfrak{h}^*$.
\end{lemma}
Вообще-то это те же рассуждения,
которые привели нас к корневому разложению самой алгебры Ли, на которой она сама действует присоединенно:
\begin{equation}
    \mathfrak{sl}_3 = \mathfrak{h}\oplus \bigoplus_{\alpha \in R} \g_{\alpha}
\end{equation}
Таким образом, веса в присоединенном представлении - это корни. Мы уже знаем, что 
$R = \{e_i - e_j| i \ne j\}$, $\g_{e_i - e_j} = \mathbb{C}E_{ij}$.

Вспомним еще, что $\mathfrak{h}=\left\{\left(\begin{array}{ccc}
h_1 & 0 & 0 \\
0 & h_2 & 0 \\
0 & 0 & h_3
\end{array}\right): h_1+h_2+h_3=0\right\}$,\\ и если рассматривать линейные функицоналы $e_i\left(\begin{array}{ccc}
h_1 & 0 & 0 \\
0 & h_2 & 0 \\
0 & 0 & h_3
\end{array}\right)=h_i$, то $\mathbb{C}(e_1+e_2+e_3) = Ann(\mathfrak{h})$, а значит 
\begin{equation}
    \mathfrak{h}^* = \mathbb{C}e_1\oplus \mathbb{C}e_2 \oplus \mathbb{C}e_3/\mathbb{C}(e_1+e_2+e_3)
\end{equation}

Пространство над $\mathbb{R}$, двойственное к алгебре Картана над $\mathbb{R}$, можно изобразить как плоскость с треугольной решеткой с базисом $\{e_1, e_2\}$. Эта решетка называется решеткой весов $\mathfrak{sl}_3$ и будет обозначаться $P$. Тогда, например, весовое разложение присоединенного представления мы изобразим так:
\begin{figure}[h]
    \centering
    \includegraphics[width = 0.5\textwidth]{lectures2/lattice1.jpeg}
    \caption{Весовые пространства присоединенного представления}
    \label{fig:enter-label}
\end{figure}

Нам также пригодится решетка с базисом $\{e_1-e_2, e_2-e_3\}$, которую мы назовем решеткой корней и обозначим $Q$.

Еще мы помним, что $ad(\g_{\alpha}):\g_{\beta} \rightarrow \g_{\alpha+\beta}$. Аналогичное конечно верно и в случае произвольного конечномерного представления.
\begin{lemma}
    Пусть $V = \bigoplus V[\lambda]$, $\lambda \in \mathfrak{h}^*$ - представление $\mathfrak{sl}_3$. Тогда $\g_{\alpha}: V[\lambda] \rightarrow V[\lambda + \alpha]$.
\end{lemma}
\begin{proof}
    Для произвольного $h \in \mathfrak{h}$, $e \in \g_{\alpha}$ $h(e.v)=e(h.v)+[h,e].v = \lambda(h)e.v+\alpha(h)e.v = (\lambda+\alpha)(h)e.v$.
\end{proof}

Как мы помним, все представления полупростых алгебр Ли вполне приводимы, поэтому наша задача как и в случае $\mathfrak{sl}_2$ сводится к изучению неприводимых представлений. Первое важное знание про них мы получим из предыдущей леммы.

\begin{corollary}
    Пусть $V = \bigoplus V[\lambda]$, $\lambda \in \mathfrak{h}^*$ - неприводимое представление $\mathfrak{sl}_3$. Тогда все его веса отличаются друг от друга на линейные комбинации корней с целыми коэффициентами.
\end{corollary}
\begin{proof}
    Допустим, это не так, и нашлись $V_{\lambda}$ и $V_{\mu}$ такие, что $\lambda$ и $\mu$ не удовлетворяют условию. Тогда выберем $v \in V_{\lambda}$ и $w \in V_{\mu}$ и подействуем на них алгеброй. Мы получим две непересекающихся прямых суммы весовых подпространств, замкнутых относительно действия алгебры, а значит $V$ не было неприводимым.
\end{proof}
\subsection*{Старший вектор}

Как и в случае $\mathfrak{sl}_2$, наша следующая задача - найти старший вектор неприводимого представления. Весовые подпространства представлений $\mathfrak{sl}_2$ нумеровались целыми числами, на которых был очевидный порядок, и мы могли сказать, что $e$ действует нулем на вектор с наибольшим весом. Как поступить в случае $\mathfrak{sl}_3$, когда на множестве весов нет очевидного порядка? Этот порядок позволит задать уже знаменитый регулярный вектор $t \in \mathfrak{h}^*$, который задавал нам поляризацию системы корней $R$.

Выберем $t \in C_+$, то есть так, что $R_+ = \{e_1-e_2, e_2-e_3, e_1 - e_3\}$. 
\begin{definition}
    Старшим весом представления $V$ мы назовем такой вес $\mu \in \mathfrak{h}^*$, что $(Re\mu, t)$ максимально среди всех остальных весов. Тогда все вектора $v \in V[\mu]$ называются старшими векторами.
\end{definition}
\begin{example}
    \begin{figure}[h!]
    \centering
    \includegraphics[width = 0.5\textwidth]{lectures2/heighest weight.jpeg}
    \caption{$\alpha$ - старший вес относительно $t \in C_+$}
    \label{fig:enter-label}
\end{figure}
\end{example}
\newpage
\begin{lemma}
    Пусть $t \in C_+$, $v \in V[\mu]$ - старший вектор. Тогда $E_{12}.v = E_{23}.v = E_{13}.v = 0$.
\end{lemma}
\begin{proof}
    $E_{ij}.v \in V[\mu + e_i-e_j]$. Но $(Re(\mu + e_i-e_j),t) > (Re(\mu), t)$ при $i < j$.
\end{proof}
\begin{theorem}
\label{verma}
    Пусть $V$ - неприводимое представление $\mathfrak{sl}_3$, $v \in V$ - старший вектор. Тогда последовательно применяя к $v$ операторы $E_{21}$, $E_{32}$, $E_{31}$ мы породим все $V$.
\end{theorem}
\begin{proof}
    Рассмотрим подпространство $W \subset V$, порожденное образами $v$ под действием 
    $E_{21}$, $E_{32}$, $E_{31}$. Поскольку $E_{31}=[E_{21}, E_{32}]$, $W$ порождена образами $v$ под действием $E_{21}$, $E_{32}$. Достаточно доказать, что $W$ замкнуто относительно действия $E_{12}$ и $E_{23}$ ($E_{13} = [E_{12}, E_{23}]$, так что замкнутость относительно $E_{13}$ последует автоматически).
    
    Пусть $w \in W$ получается из $v$ применением $m$ раз генератора $E_{21}$ и $l$ раз генератора $E_{32}$ в каком-то порядке. Тогда $w \in V[\mu + \alpha]$, $\alpha = m(e_2-e_1)+l(e_3-e_2)$. Доказательство будем вести индукцией по высоте $\alpha$ (для краткости будем говорить по высоте $w$). По определению $E_{12}.v = E_{23}.v = 0$. Теперь пусть $w$ таков, что высота $w$ равна $n$. Тогда возможны два случая: $w = E_{21}.u$, $w = E_{32}.u$, где в обоих случаях высота $u$ равна $n-1$.
    $$E_{12}(E_{21}.u) = E_{21}(E_{12}.u)+[E_{12}, E_{21}].u$$
    Первое слагаемое лежит в $W$ по предположению индукции, а второе потому, что $[E_{12}, E_{21}]$ лежит в $\mathfrak{h}$, а значит действует на $u$ диагонально.
    $$E_{23}(E_{21}.u) = E_{21}(E_{23}.u)$$
    Первое (и единственное слагаемое, $[E_{23}, E_{21}]=0$) лежит в $W$ по предположению индукции.
    Аналогичная проверка проводится для $w=E_{32}.u$.
\end{proof}

\begin{corollary}
    Подпространство $V[\mu]$ одномерно, а $V[\mu + m(e_2-e_1)]$ и $V[\mu + l(e_3-e_2)]$ не более чем одномерны. 
\end{corollary}
\begin{remark}
Веса неприводимого представления расположены в $1/3$-плоскости с вершиной в $\mu$.
\begin{figure}[h!]
    \centering
    \includegraphics[width = 0.5\textwidth]{weights.jpeg}
    \caption{1/3 плоскости}
    \label{fig:enter-label}
\end{figure}
\end{remark}
\begin{corollary}
    Пусть $V$ представление $\mathfrak{sl}_3$, $v$-старший вектор. Тогда подпредставление $W$, порожденное образами $v$ под действием $E_{21}$, $E_{32}$, $E_{31}$ неприводимо.
\end{corollary}
\begin{proof}
    Пусть $\mu$ - вес $v$. Тогда если $W=W'\oplus W''$ (мы знаем, что представления полупростых алгебр Ли вполне приводимы), то $W[\mu] = W'[\mu]\oplus W''[\mu]$, поскольку действие $\mathfrak{h}$ и проекции на $W'$ и $W''$ коммутируют. Но из сказанного выше следует, что $W[\mu]$ одномерно, значит или $W'[\mu]$, или $W''[\mu]$ нулевое. Это значит что нулевым будет $W'$ или $W''$.
\end{proof}
Из этого следствия в частности следует, что старший вес неприводимого представления единственный. В самом деле, если имеется еще один старший вес $\mu'$, то рассмотрим $w \in V[\mu']$ и подпредставление $W$, порожденное образами $w$ под действием $E_{21}$, $E_{32}$, $E_{31}$. По следствию $W$ неприводимо, но оно не совпадает с $V$, так что $W=0$.
\subsection*{Весовые подпространства неприводимого представления}
Посмотрим внимательно на подпространства, соответствующие граничным весам:
\begin{equation}
    V'=\bigoplus_{k \in \mathbb{Z}_+} V[\mu + k(e_2-e_1)], \quad V''=\bigoplus_{k \in \mathbb{Z}_+} V[\mu + k(e_3-e_2)]
\end{equation}

Они граничные в том смысле, что $E_{23}.V'=E_{13}.V'=0$, $E_{12}.V''=E_{13}.V''=0$. $V'$ замкнута относительно $\mathfrak{sl}_2$-тройки $\{E_{12}, H_{12}, E_{21}\}$, где $H_{12} = h_{e_1 - e_2} = [E_{12}, E_{21}]$. Более того, это представление старшего веса $\mu(H_{12})=(e_1-e_2, \mu)$. Оно конечномерно только если $(e_1-e_2, \mu) \in \mathbb{Z}_+$.
Аналогичное рассуждение для $V''$ приводит к условию $(e_2-e_3, \mu) \in \mathbb{Z}_+$. Поэтому конечномерность представления $V$ гарантирует нам, что старший вес лежит на решетке весов.

Взглянем на $V'$ еще раз. Это неприводимое конечномерное представление $\mathfrak{sl}_2$, значит его веса симметричны относительно 0. Это значит, что множество точек, соответствующих весам $V'$, на решетке весов симметрично относительно прямой $\{\lambda \in \mathfrak{h}^*|(\lambda, e_1-e_2)=0\}=L_{e_1-e_2}$. 

Пусть теперь $\mu'$ - вес, симметричный относительно этой прямой. Он младший относительно представления $\mathfrak{sl}_2$-тройки $\{E_{12}, H_{12}, E_{21}\}$, то есть для 
$v' \in V[\mu']$ $E_{21}.v' = 0$. Но еще $E_{13}.v' = 0$ и $E_{23}.v' = 0$. Это значит, что если теперь в качестве простых корней мы возьмем $e_1 - e_3$ и $e_2 - e_1$, $v'$ будет старшим вектором.
\begin{remark}
    Это согласуется с тем, что $\mu'$ лежит в камере Вейля $s_{e_1-e_2}(C_+)$, симметричной $C_+$ относительно стенки $L_{e_1-e_2}$, то есть вектор $v'$ является старшим относительно $t'$ из камеры Вейля $s_{e_1-e_2}(C_+)$. Простые корни для этой поляризации - это $e_1 - e_3$ и $e_2 - e_1$.
\end{remark}

Повторяя рассуждения для новой поляризации, получаем, что набор весов $\{\mu' + k(e_3-e_1)|k\in \mathbb{Z}_+\}$ симметричен относительно прямой $L_{e_1-e_3}$. Повторяя рассуждения для каждой следующей камеры Вейля, мы выясняем, что все веса представления заключены в шестиугольнике, вершины которого получаются из $\mu$ отражениями относительно стенок камер Вейля $L_{e_i - e_j}$.

Нам осталось понять, какие из весов внутри шестиугольника присутствуют в весовом разложении. По теореме $\ref{verma}$ в него могут входить только веса, конгруэнтные $\mu$, то есть такие $\lambda$, что $\lambda - \mu$ лежит на решетке корней. На самом деле каждый такой вес входит в разложение с ненулевой кратностью: возьмем любой вес $\beta$ на одном из ребер шестиугольника, выберем $w \in V[\beta]$ и подействуем на него $E_{ij}$ (выберем $E_{ij}$ так, чтобы вес $\beta + e_i - e_j$ был внутри шестиугольника). Мы получим неприводимое представление $\mathfrak{sl}_2$-тройки $\{E_{ij}, H_{ij}, E_{ji}\}$, веса которого расположены симметрично относительно прямой $L_{e_i-e_j}$. Это значит, что все слагаемые прямой суммы $\bigoplus_{k \in \mathbb{Z}_+}V[\beta + k(e_i-e_j)]$ ненулевые.

Подведем итог.
\begin{theorem}
    Пусть $V$ - неприводимое конечномерное представление $\mathfrak{sl}_3$. Тогда для некоторого $\mu$, лежащего на решетке весов, множество весов представления $V$ - это такие $\lambda \in P$, что они конгруэнтны $\mu$ и лежат внутри шестиугольника с вершинами, полученными отражениями относительно прямых $L_{e_i-e_j}$.
\end{theorem}
\begin{example}
    \begin{figure}[h!]
    \centering
    \includegraphics[width = 0.5\textwidth]{lectures2/weightsofrep.jpeg}
    \caption{набор весов неприводимого представления со страшим весом $\alpha$}
    \label{fig:enter-label}
    \end{figure}
\end{example}
\end{document}